\chapter{时域积分方程基础}

时域积分方程(TDIE)方法作为分析瞬态电磁波动现象最主要的数值算法之一,常用于求解均匀散射体和表面散射体的瞬态电磁散射问题。

\section{时域积分方程的类型}

\section{空间基函数与时间基函数}

利用数值算法求解时域积分方程,首先需要选取适当的空间基函数与时间基函数对待求感应电流进行离散。

\subsection{空间基函数}

RWG基函数是定义在三角形单元上的最具代表性的基函数。它的具体定义如下:
\begin{equation}
f_n(r)=
\begin{cases}
\frac{l_n}{2A_n^+}\rho_n^+=\frac{l_n}{2A_n^+}(r-r_+)&r\in T_n^+\\
\frac{l_n}{2A_n^-}\rho_n^-=\frac{l_n}{2A_n^-}(r_--r)&r\in T_n^-\\
0&\text{其它}\\
\end{cases}
\end{equation}
其中,$l_n$为三角形单元$T_n^+$和$T_n^-$公共边的长度,$A_n^+$和$A_n^-$分别为三角形单元$T_n^+$和$T_n^-$的面积(如图~\ref{fig:rwg-demo} 所示)。

\begin{Figure}{fig:rwg-demo}{RWG基本函数几何参数示意图}
  \includegraphics[width=.5\textwidth]{ch/02/rwg-demo}
\end{Figure}

\subsection{时间基本函数}

\ldots

\subsubsection{时域方法特有的展开函数}

\ldots

\subsubsection{频域方法特有的展开函数}

\ldots

\subsection{入射波}

如图~\ref{fig:wave-a} 和图~\ref{fig:wave-b} 所示分别给出了参数$E_0=\hat{x}$,$a_n=-\hat{z}$,$f_0=250MHz$,$f_w=50MHz$,$t_w=4.2\sigma$时,调制高斯脉冲的时域与频域归一化波形图。

\begin{Figure}{fig:wave}{调制高斯脉冲时域与频率波形\\
      (a) 调制高斯脉冲时域波形;(b) 调制高斯脉冲频域波形}
  \subfloat[]{
    \label{fig:wave-a}
    \includegraphics[width=0.4\textwidth]{ch/02/wave-a}}
  \subfloat[]{
    \label{fig:wave-b}
    \includegraphics[width=0.4\textwidth]{ch/02/wave-b}}
\end{Figure}

\section{本章小结}

本章首先从时域麦克斯韦方程组出发推导得到了时域电场、磁场以及混合场积分方程。
